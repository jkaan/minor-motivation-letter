\documentclass{letter}
\signature{Joey Kaan}
\address{Joey Kaan \\ 64808 \\ HBO-ICT}
\longindentation=0pt

\begin{document}

\pagenumbering{gobble} % Passes page numbering to \gobble which eats its arguments. So no output is returned.

\begin{letter}{Hogeschool Zeeland \\ T.a.v. examencomissie \\ Edisonweg 4 \\ 4382 NW Vlissingen}

\opening{Betreft: Inschrijving externe minor \newline \\ Beste heer/mevrouw,}

Deze brief gaat over mijn motivatie voor het kiezen van mijn minor.

Als eerste; de gekozen minor heet App Development. De reden dat ik deze minor gekozen heb is, omdat ik altijd al heb willen weten hoe een app gemaakt wordt. Tegenwoordig worden er ook veel apps gemaakt en veel bedrijven zijn ook op zoek naar werknemers met dergelijke vaardigheden. Bij een app maken komt natuurlijk veel meer kijken dan alleen het programmeren en tijdens deze minor wordt daar ook aandacht aan geschonken.

\newline

Tijdens de minor wordt er gewerkt in een groep waarbij iedereen een ander veld van expertise zal hebben, voor mijn eigen vaardigheden vind ik het heel erg belangrijk om te zien hoe andere mensen bepaalde problemen aanpakken, want ik kan hier ook veel van leren. Het is heel gemakkelijk om te denken dat jouw manier de beste is, maar het is belangrijk om dit te vergelijken met andere mensen. 

De projectgroep bepaalt ook zelf welke kant er op gegaan wordt met het project. Vragen zoals welke ontwikkelmethodiek kiezen we, hoe gaan we dit project aanpakken en hoe we bepaalde zaken op gaan lossen zullen aan bod komen en hier zal in de projectgroep een antwoord op gevonden moeten worden. Er wordt vanuit de minor geen houvast gegeven, de projectgroep is volledig verantwoordelijk voor het proces en het resultaat. Hierbij werken we dus niet alleen aan onze vakinhoudelijke vaardigheden, maar ook aan onze persoonlijk, organisatorische en communicatieve vaardigheden. Hoewel we aanwezig zullen zijn op het bedrijf bij de opdrachtgever moet het worden gezien als een extern ingehuurde instantie, omdat wij zelf alle keuzes maken. 

Hiernaast blijven we als team up-to-date door workshops en lessen te volgen aan de Christelijke Hogeschool Windesheim.

\newline

Wat betreft de overlap met opleiding: in het onderwijs- en examenreglement (OER) van de opleiding HBO-ICT komt in het programma geen cursus voor wat te zich richt op het ontwikkelen van een native App, dit betekent een applicatie maken met een taal die specifiek gericht is op het platform zoals bijvoorbeeld iOS en Android. De focus in deze minor ligt op het complete ontwikkel traject van het realiseren van een native App. We zullen in deze minor in aanraking komen met nieuwe technieken, denk aan het ontwikkelen van een app voor apple iOS wat niet vrij toegankelijk is. 

\newline

De opleiding HBO-ICT richt zich voornamelijk op het ontwikkelen voor web gebruik in plaats van mobiel gebruik. Mobiel is op dit moment heel erg populair en veel bedrijven spelen hier op in. 
De mogelijkheden zijn oneindig daarom is het belangrijk om te leren hoe een app wordt ontwikkeld en wat hier allemaal bij komt kijken.

\newline

Een uitdaging en een goede kans om mezelf daar in te verdiepen en wellicht een goede naam achter te laten bij het bedrijf waarvoor de App ontwikkeld gaat worden. 

\newline

Overigens zitten alle beroepscompetenties in deze minor zoals ze beschreven staan in het onderwijs- en examenreglement van de opleiding HBO-ICT.

\newline

Tot slot, gedurende de minor wordt er natuurlijk gewerkt in en voor een bedrijf. Hierdoor wordt er dus ervaring opgedaan met het werken in een professionele omgeving wat heel erg belangrijk is, want werkervaring wordt vanuit een bedrijfs perspectief als zeer belangrijk ervaren. Ook wordt er gewerkt om echte problemen van een echt bedrijf op te lossen. Dit maakt de minor tot een uiterst leerzame ervaring, omdat je leert vanuit een echt probleem een app te maken en die gebruikt zal worden door een bedrijf.


\closing{Met vriendelijke groet, }
\ps

\end{letter}

\end{document}
